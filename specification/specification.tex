% Copyright (C) 2010  Martin Lambers <marlam@marlam.de>
%

\documentclass[a4paper,11pt]{article}
\usepackage[english]{babel}
\usepackage{mathptmx}
\usepackage[utf8]{inputenc}
\usepackage[T1]{fontenc}
\usepackage{hyperref}

\pagestyle{empty}

\newcommand{\code}[1]{\texttt{#1}}

\begin{document}

\title{The Generic Tagged Array File Format}
%\author{Martin Lambers}
\author{~}
%\date{}%
\date{~}
\maketitle

\noindent
{\scriptsize
Copyright (C) 2010 Martin Lambers\\~\\
Permission is granted to copy, distribute and/or modify this document
under the terms of the GNU Free Documentation License, Version 1.2 or
any later version published by the Free Software Foundation; with no
Invariant Sections, no Front-Cover Texts, and no Back-Cover Texts.
%A copy of the license is included in the section entitled ``GNU Free
%Documentation License''.
}


\phantomsection 
\addcontentsline{toc}{section}{Changes}
\section*{Changes}

\begin{itemize}
\item \textbf{2010-06-06} Various extensions, cleanups, clarifications.
The list of reserved tags was moved to an external file because it may change
while the file format itself remains unchanged.
\item \textbf{2010-04-16} Initial version.
\end{itemize}


\phantomsection 
\addcontentsline{toc}{section}{Introduction}
\section*{Introduction}

This document specifies version 1 of the Generic Tagged Array (GTA) file format.
This file format has the following features:
\begin{itemize}
\item GTA files can store any kind of data in multidimensional arrays
\item GTA files can use simple tags to describe the data
\item GTA files are streamable, which allows direct reading from and
	writing to pipes, network sockets, or other non-seekable media
\item GTA files can optionally use ZLIB, BZIP2, or XZ compression,
	allowing a tradeoff between compression/decompression speed and
	compression ratio
\item Uncompressed GTA files allow easy out-of-core data access
\end{itemize}


\section{Definitions}

\subsection{Arrays}

An array has zero or more \emph{dimensions}, and a size of at least one in each
dimension. These dimension sizes determine the number of array elements.
An array element consists of zero or more \emph{components}.
Each component has a \emph{type}. The types are defined in table~\ref{tab:types}.
The special type \code{GTA\_BLOB} allows the user to define custom types
of arbitrary length.

\begin{table}
\begin{tabular}{l|r|r|p{.55\textwidth}}
Type & No. & Bytes & Description\\\hline
\code{GTA\_INT8}      &  1 & 1		& C99 \code{int8\_t}		\\
\code{GTA\_UINT8}     &  2 & 1 		& C99 \code{uint8\_t}		\\
\code{GTA\_INT16}     &  3 & 2 		& C99 \code{int16\_t}		\\
\code{GTA\_UINT16}    &  4 & 2 		& C99 \code{uint16\_t} 	\\
\code{GTA\_INT32}     &  5 & 4 		& C99 \code{int32\_t}		\\
\code{GTA\_UINT32}    &  6 & 4 		& C99 \code{uint32\_t}		\\
\code{GTA\_INT64}     &  7 & 8 		& C99 \code{int64\_t} 		\\
\code{GTA\_UINT64}    &  8 & 8 		& C99 \code{uint64\_t}		\\
\code{GTA\_INT128}    &  9 & 16		& C99 \code{int128\_t} (unavailable on many platforms)		\\
\code{GTA\_UINT128}   & 10 & 16		& C99 \code{uint128\_t} (unavailable on many platforms)		\\
\code{GTA\_FLOAT32}   & 11 & 4 		& IEEE 754 single precision floating point (on many platforms: \code{float}) \\
\code{GTA\_FLOAT64}   & 12 & 8 		& IEEE 754 double precision floating point (on many platforms: \code{double}) \\
\code{GTA\_FLOAT128}  & 13 & 16		& IEEE 754 quadrupel precision floating point (unavailable on many platforms, even if \code{long double} exists) \\
\code{GTA\_CFLOAT32}  & 14 & 8		& complex (re,im) based on \code{GTA\_FLOAT32} \\
\code{GTA\_CFLOAT64}  & 15 & 16		& complex (re,im) based on \code{GTA\_FLOAT64} \\
\code{GTA\_CFLOAT128} & 16 & 32		& complex (re,im) based on \code{GTA\_FLOAT128} \\
\code{GTA\_BLOB}      &  0 & ---    	& Data blob; must be endianness-independent; user must specify the size \\
\end{tabular}
\caption{GTA data types. All integer types use the common two-complement
representation. All floating point types conform to IEEE 754. A corresponding
C99 type might not exist on all platforms.}
\label{tab:types}
\end{table}

For example, an image with $640\times 480$ pixels in the common RGB format
would have two dimensions, the first with size 640, the second with size 480.
Each array element would consists of three components, all of type
\code{GTA\_UINT8}.


\subsection{Tags}
\label{sec:tags}

A \emph{tag} consists of a tag name and a tag value.

A tag name is a non-empty, zero-terminated, \mbox{UTF-8} encoded string that does not
contain control characters (characters with a code less than 32 or equal to 127)
and does not contain the character \code{=}.

A tag value is a zero-terminated, \mbox{UTF-8} encoded string that does not contain
control characters (characters with a code less than 32 or equal to 127). It can
be empty, and it can contain the character \code{=}.

Leading or trailing white space is not ignored; it is part of the name or
value string.

A \emph{tag list} consists of zero or more tags.

An array has multiple tag lists:
\begin{itemize}
\item One global tag list, containing tags that are relevant to the array as a
whole.
\item A dimension tag list for each dimension, containing tags that are
relevant to this particular dimension.
\item A component tag list for each element component, containing tags that
are relevant to this particular component.
\end{itemize}

An array does not need to have any tags in its tag lists. If it does have
tags, these are subject to certain rules, see the external documentation about
reserved tags.

For example, an image might have a global tag list containing the tag
\mbox{\code{DATE=Fri,~~4 Dec 2009 22:29:43 +0100 (CET)}}, 
a tag list for the first dimension containing the tag \code{INTERPRETATION=X},
a tag list for the second dimensions containing the tag
\code{INTERPRETATION=Y}, and three component tag lists, containing the tags
\code{INTERPRETATION=SRGB/RED}, \code{INTERPRETATION=SRGB/GREEN},
\code{INTERPRETATION=SRGB/BLUE} respectively.


\subsection{Chunks}

A \emph{chunk} stores a number of chunk data bytes, either compressed or uncompressed.
It begins with a chunk header, defined as follows:
\begin{itemize}
\item The first 8 bytes of the chunk header contain a value of type \code{GTA\_UINT64},
subject to the endianness type of the file (see next section for details).
This value is the number of bytes of the chunk data (excluding the
header). Currently, it is limited to $2^24$, so that a chunk cannot store more
than 16 MiB.
\item Only if the previous value is not zero, the next byte of the chunk
contains a value of type \code{GTA\_UINT8}.  This value specifies the
compression method of the chunk, as defined in Tab.~\ref{tab:compression}.
\item Only if the compression method is not \code{GTA\_NONE}, the next 8 bytes
contain a value of type \code{GTA\_UINT64}, subject to the endianness type of
the file. This value is the number of bytes that the compressed chunk data
uses. It is guaranteed to be less than or equal to the number of bytes of the
chunk data. If a compression method would produce compressed chunk data that
is larger than the uncompressed chunk data, the chunk is simply not compressed
but stored uncompressed (with the compression method \code{GTA\_NONE}).
\end{itemize}

\begin{table}
\begin{tabular}{l|l|l}
Compression method & Number & Description\\\hline
\code{GTA\_NONE}  & 0 & No compression \\
\code{GTA\_ZLIB}  & 1 & ZLIB compression with default level (fast,\\
                  &   & moderate compression ratio) \\
\code{GTA\_ZLIB1} & 4 & ZLIB compression with level 1\\
\code{GTA\_ZLIB2} & 5 & ZLIB compression with level 2\\
\code{GTA\_ZLIB3} & 6 & ZLIB compression with level 3\\
\code{GTA\_ZLIB4} & 7 & ZLIB compression with level 4\\
\code{GTA\_ZLIB5} & 8 & ZLIB compression with level 5\\
\code{GTA\_ZLIB6} & 9 & ZLIB compression with level 6\\
\code{GTA\_ZLIB7} & 10& ZLIB compression with level 7\\
\code{GTA\_ZLIB8} & 11& ZLIB compression with level 8\\
\code{GTA\_ZLIB9} & 12& ZLIB compression with level 9\\
\code{GTA\_BZIP2} & 2 & BZIP2 compression (moderate speed,\\
                  &   & good compression ratio) \\
\code{GTA\_XZ}    & 3 & XZ compression (low compression speed,\\
                  &   & moderate decompression speed,\\
		  &   & good or very good compression rates) \\
\end{tabular}
\caption{GTA compression methods.}
\label{tab:compression}
\end{table}

After the chunk header (8, 9, or 17 bytes, depending on the chunk data size
and compression method), the chunk data follows, either uncompressed, using
the number of bytes given in the first header value, or compressed, using
the number of bytes given in the third header value.

A \emph{chunk list} is a list of one or more chunks. A chunk with the chunk
data size zero marks the last chunk in a chunk list.


\section{File Format}

A GTA consists of the GTA header and the GTA data. The GTA header stores
information about the file format, the list of dimensions, the list of element
components, and all tag lists. See Sec.~\ref{sec:gta-header}.
The GTA data is stored immediately after the header. See Sec.~\ref{sec:gta-data}.


\subsection{GTA Header}
\label{sec:gta-header}

The first three bytes of a GTA Header are of type \code{GTA\_UINT8} and contain
the \mbox{UTF-8} codes of the characters \code{G}, \code{T}, and \code{A}.

The fourth byte is of type \code{GTA\_UINT8} and contains the GTA file format
version number. Currently, this value must always be 1.

The fifth byte is of type \code{GTA\_UINT8}, and each bit represents a flag that
can be set (1) or unset (0).

The lowest bit flag determines if multibyte values in the file are in big endian
(1) or little endian (0) byte ordering. This flag applies to all values in the
header and in the data. (The array element component type \code{GTA\_BLOB} is special:
it must be independent of byte ordering, so that it is valid regardless of this
flag.)

The second lowest bit flag determines if the GTA array data that follows the header 
is stored in chunks (1) or not (0).

All six remaining bit flags are reserved and must currently always be unset (0).

The sixth byte in the header is of type \code{GTA\_UINT8} and contains a hint on
the compression type of the GTA. It must contain one of the values defined in
Tab.~\ref{tab:compression}.
If this value if \code{GTA\_NONE}, then the GTA data must not be stored in chunks.
If this value is not \code{GTA\_NONE}, then the GTA data must be stored in
chunks. In this case, each chunk defines its own compression method, which may
differ from the method indicated by this value.
An application can store the users choice of compression method in this hint
value, so that other tools can reproduce the compression after altering parts of
a GTA.

At byte 7 of the header, a chunk list starts. This chunk list contains the
information described in the next paragraphs.

The fist information in the chunk list is the list of element components. It
consists of zero or more component definitions. A component definition starts
with one byte of type \code{GTA\_UINT8} that contains one of the values
defined in Tab.~\ref{tab:types}. If the type is \code{GTA\_BLOB}, then the
component definition contains additional 8 bytes of type \code{GTA\_UINT64}
(subject to the endianness type of the GTA). This additional value stores the
size in bytes of the component. A component definition that contains the
special value 255 in its type field signals the end of the component list.

After the list of element components, the list of dimensions is stored.  This
list consists of zero or more dimension definitions. A dimension definition
consists of 8 bytes of type \code{GTA\_UINT64} (subject to the endianness type
of the GTA). This value stores the size of the dimension, and must be greater
than zero. A dimension definition that stores the special value 0 in its size
field signals the end of the dimension list.

After the list of dimensions, the global tag list is stored.  The global tag
list stores each tag name and its tag value as-is (\mbox{UTF-8} strings, zero
terminated). An empty name (consisting only of the value 0) signals
the end of the global tag list.

After the global tag list, the tag lists of all element components follows, and
after that the tag lists of all dimensions. These tag lists are stored in the
same manner as the global tag list.

After these tag list definitions, the GTA header chunk list ends, i.e. a
single zero-sized chunk must signal the end of the header chunk list.

The GTA array data follows immediately after the header.


\subsection{GTA Data}
\label{sec:gta-data}

The array data itself is stored in a compact form without alignment offsets or
other fill bytes of any kind. The linear format of a multidimensional array is
the same as that of a C array when the first dimension is specified last
(row-major for two-dimensional data). The following is an example for a
three-dimensional array: \code{element\_type array[size2][size1][size0]}.

The array data is either stored in a chunk list (if the corresponding flag in
the header is set) or as-is.

When stored as-is, individual data elements can easily be accessed if the input
is seekable.

Immediately after the array data, another GTA may follow.



%\newpage
%\renewcommand{\large}{\small}
%\renewcommand{\Large}{\normalsize}
%\newcommand{\chapter}[3][\relax]{\section*{#3}\footnotesize}
%\include{fdl}

\end{document}
